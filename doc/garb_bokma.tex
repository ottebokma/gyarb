% gMaster LaTex file for gyarb

% Shortcuts
% C-c C-c compile

\documentclass[11pt]{article}

% packages
\usepackage[backend=biber]{biblatex}
\usepackage{graphicx}
\usepackage{wrapfig}

% settings
\addbibresource{garb_bibl/garb_bibl.bib}

% title
\title{Gymnasiearbete {\LaTeX} dokument}
\author{Otte Bokma}

% document
\begin{document}

\begin{titlepage}

  \begin{center}
    \huge{Prediction of long term effects of TBI with machine learning}
    \\[1cm]
    \large{Using machine learning to classify long term effects of traumatic brain injury}
  \end{center}
\end{titlepage}

\section*{Sammanfattning}

\subsection*{Bakgrund}

Hjärnskakningar är skador som är mer eller mindre vanliga beroende på en persons dagliga aktiviteter. De framkommer framför allt bland idrottare inom vissa sporter, emepelvis boxning, amerikansk fotboll och ishockey. Hjänskakningar är även en relativt vanlig skada vid trafikolyckor (whiplash). Det finns dock väldigt många olika former av hjänskakningar och symptomen varierar på både lång och kort sikt väldigt mycket mellan fallen. Att förutsäga de långvariga symptomen kan vara väldigt svårt. En del personer påverkas nästan inte medan andra kan få problem som varar i många år.\\
\\
Artificiell intelligens är något som används väldigt mycket i nuläget för en mängd olika saker. Artificiell intelligens har visat sig kunna användas väldigt effektivt för bland annat denna typ av uppgift, att identifiera eller förutsäga något utifrån olika variabler. Det skulle alltså vara tänkbart att artificiell intelligens till viss mån skulle kunna identifiera möjliga långvariga symptom av en hjärnskakning utifrån de kortvariga symptomen.

\subsection*{Metod och Resultat}

\subsection*{Slutsats}

\section*{Abstract}

\subsection*{Background}

Traumatic Brain injury (TBI) is a condition with varying `` depending on a persons daily activities. Above all TBI shows among athletes of some sports, such as boxing, american football and icehockey. TBI is also a relatively common injury in traffic accidents(, often in connection with a whiplash type injury). There are many different types of TBI and the sypmtoms can vary widely in both long term and short term. To predict the long term effects of TBI can therefore be very hard. Som people have almost no long term symptoms whilst others may have sypmtoms years after the accident occured.\\
\\
Nowadays artificial intelligence is used widely and has many different kinds of application. AI has been shown to be very suitable for this type of application , to identify or pedict something based on a set of variables. Thus it could be conceivable that AI could to at least some extent identify possible long term effects based on the short term symptoms of a TBI patient.

*
Traumatic brain injury (TBI) is a condition that can be quite hard to diagnose and even if it is diagnosed the long term effects and symptoms can be hard to identify. Machine learning and artificial intelligence has in recent times gained popularity as is gives a way of identifying and predicting things that humans can not always do. As for TBI there is a lot of data to be collected about a patients health status at the time of being examined. This data is unfortunately quite hard for humans to interpret. Using machine learning to interpret this data is a great opportunity to harness the power of data classification that is machine learning.
*

\subsection*{Method and Findings}

\subsection*{Conclusion}

\tableofcontents

\section{Introduction}

\subsection{Purpose}

\subsection{``Frågeställning''}

\section{Method}

\section{Results}

\section{Discussion}

\begin{figure}[h]
  \centering
  \includegraphics[width=5cm]{graphics/ubuntu.png}
  \caption{Ubuntu logo}
\end{figure}

As \cite{FeatureSelection2020} bla, bla, bla...\\

\section{Bibliography}

\printbibliography

\end{document}