% Master LaTex file for gyarb

% Shortcuts
% C-c C-c compile

\documentclass[11pt]{article}

% packages
\usepackage[backend=biber]{biblatex}
\usepackage{graphicx}
\usepackage{wrapfig}

% settings
\addbibresource{garb_bibl/garb_bibl.bib}

% title
\title{Gymnasiearbete {\LaTeX} dokument}
\author{Otte Bokma}

% document
\begin{document}

\begin{titlepage}

  \begin{center}
    \huge{Prediction of long term effects of TBI with machine learning}
    \\[1cm]
    \large{Using machine learning for classifying long term effects of traumatic brain injury}
  \end{center}
\end{titlepage}

\section*{Sammanfattning}

\subsection*{Bakgrund}

\subsection*{Metod och Resultat}

\subsection*{Slutsats}

\section*{Abstract}

\subsection*{Background}

*
Traumatic brain injury (TBI) is a condition that can be quite hard to diagnose and even if it is diagnosed the long term effects and symptoms can be hard to identify. Machine learning and artificial intelligence has in recent times gained popularity as is gives a way of identifying and predicting things that humans can not always do. As for TBI there is a lot of data to be collected about a patients health status at the time of being examined. This data is unfortunately quite hard for humans to interpret. Using machine learning to interpret this data is a great opportunity to harness the power of data classification that is machine learning.
*

\subsection*{Method and Findings}

\subsection*{Conclusion}

\tableofcontents

\section{Introcuction}

\subsection{Purpose}

\subsection{Issue}

\section{Method}

\section{Results}

\begin{figure}[h]
  \centering
  \includegraphics[width=5cm]{graphics/ubuntu.png}
  \caption{Ubuntu logo}
\end{figure}

As \cite{FeatureSelection2020} bla, bla, bla...\\

\section{Bibliography}

\printbibliography

\end{document}